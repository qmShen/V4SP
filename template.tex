\documentclass[journal]{vgtc}                % final (journal style)
%\documentclass[review,journal]{vgtc}         % review (journal style)
%\documentclass[widereview]{vgtc}             % wide-spaced review
%\documentclass[preprint,journal]{vgtc}       % preprint (journal style)

%% Uncomment one of the lines above depending on where your paper is
%% in the conference process. ``review'' and ``widereview'' are for review
%% submission, ``preprint'' is for pre-publication, and the final version
%% doesn't use a specific qualifier.

%% Please use one of the ``review'' options in combination with the
%% assigned online id (see below) ONLY if your paper uses a double blind
%% review process. Some conferences, like IEEE Vis and InfoVis, have NOT
%% in the past.

%% Please use the ``preprint''  option when producing a preprint version
%% for sharing your article on an open access repository

%% Please note that the use of figures other than the optional teaser is not permitted on the first page
%% of the journal version.  Figures should begin on the second page and be
%% in CMYK or Grey scale format, otherwise, colour shifting may occur
%% during the printing process.  Papers submitted with figures other than the optional teaser on the
%% first page will be refused. Also, the teaser figure should only have the
%% width of the abstract as the template enforces it.

%% These few lines make a distinction between latex and pdflatex calls and they
%% bring in essential packages for graphics and font handling.
%% Note that due to the \DeclareGraphicsExtensions{} call it is no longer necessary
%% to provide the the path and extension of a graphics file:
%% \includegraphics{diamondrule} is completely sufficient.
%%
\ifpdf%                                % if we use pdflatex
  \pdfoutput=1\relax                   % create PDFs from pdfLaTeX
  \pdfcompresslevel=9                  % PDF Compression
  \pdfoptionpdfminorversion=7          % create PDF 1.7
  \ExecuteOptions{pdftex}
  \usepackage{graphicx}                % allow us to embed graphics files
  \DeclareGraphicsExtensions{.pdf,.png,.jpg,.jpeg} % for pdflatex we expect .pdf, .png, or .jpg files
\else%                                 % else we use pure latex
  \ExecuteOptions{dvips}
  \usepackage{graphicx}                % allow us to embed graphics files
  \DeclareGraphicsExtensions{.eps}     % for pure latex we expect eps files
\fi%

%% it is recomended to use ``\autoref{sec:bla}'' instead of ``Fig.~\ref{sec:bla}''
\graphicspath{{figures/}{pictures/}{images/}{./}} % where to search for the images

\usepackage{microtype}                 % use micro-typography (slightly more compact, better to read)
\PassOptionsToPackage{warn}{textcomp}  % to address font issues with \textrightarrow
\usepackage{textcomp}                  % use better special symbols
\usepackage{mathptmx}                  % use matching math font
\usepackage{times}                     % we use Times as the main font
\renewcommand*\ttdefault{txtt}         % a nicer typewriter font
\usepackage{cite}                      % needed to automatically sort the references
\usepackage{tabu}                      % only used for the table example
\usepackage{booktabs}                  % only used for the table example
%% We encourage the use of mathptmx for consistent usage of times font
%% throughout the proceedings. However, if you encounter conflicts
%% with other math-related packages, you may want to disable it.

%% In preprint mode you may define your own headline. If not, the default IEEE copyright message will appear in preprint mode.
%\preprinttext{To appear in IEEE Transactions on Visualization and Computer Graphics.}

%% In preprint mode, this adds a link to the version of the paper on IEEEXplore
%% Uncomment this line when you produce a preprint version of the article 
%% after the article receives a DOI for the paper from IEEE
%\ieeedoi{xx.xxxx/TVCG.201x.xxxxxxx}

%% If you are submitting a paper to a conference for review with a double
%% blind reviewing process, please replace the value ``0'' below with your
%% OnlineID. Otherwise, you may safely leave it at ``0''.
\onlineid{0}

%% declare the category of your paper, only shown in review mode
\vgtccategory{Research}
%% please declare the paper type of your paper to help reviewers, only shown in review mode
%% choices:
%% * algorithm/technique
%% * application/design study
%% * evaluation
%% * system
%% * theory/model
\vgtcpapertype{please specify}

%% Paper title.
\title{Global Illumination for Fun and Profit}

%% This is how authors are specified in the journal style

%% indicate IEEE Member or Student Member in form indicated below
% \author{Roy G. Biv, Ed Grimley, \textit{Member, IEEE}, and Martha Stewart}
% \authorfooter{
% %% insert punctuation at end of each item
% \item
%  Roy G. Biv is with Starbucks Research. E-mail: roy.g.biv@aol.com.
% \item
%  Ed Grimley is with Grimley Widgets, Inc.. E-mail: ed.grimley@aol.com.
% \item
%  Martha Stewart is with Martha Stewart Enterprises at Microsoft
%  Research. E-mail: martha.stewart@marthastewart.com.
% }

%other entries to be set up for journal
% \shortauthortitle{Biv \MakeLowercase{\textit{et al.}}: Global Illumination for Fun and Profit}
%\shortauthortitle{Firstauthor \MakeLowercase{\textit{et al.}}: Paper Title}

%% Abstract section.
\abstract{Duis autem vel eum iriure dolor in hendrerit in vulputate
velit esse molestie consequat, vel illum dolore eu feugiat nulla
facilisis at vero eros et accumsan et iusto odio dignissim qui blandit
praesent luptatum zzril delenit augue duis dolore te feugait nulla
facilisi. Lorem ipsum dolor sit amet, consectetuer adipiscing elit,
sed diam nonummy nibh euismod tincidunt ut laoreet dolore magna
aliquam erat volutpat. Ut wisi enim ad minim veniam, quis nostrud exerci tation ullamcorper
suscipit lobortis nisl ut aliquip ex ea commodo consequat. Duis autem
vel eum iriure dolor in hendrerit in vulputate velit esse molestie
consequat, vel illum dolore eu feugiat nulla facilisis at vero eros et
accumsan et iusto odio dignissim qui blandit praesent luptatum zzril
delenit augue duis dolore te feugait nulla facilisi.%
} % end of abstract

%% Keywords that describe your work. Will show as 'Index Terms' in journal
%% please capitalize first letter and insert punctuation after last keyword
\keywords{Radiosity, global illumination, constant time}

%% ACM Computing Classification System (CCS). 
%% See <http://www.acm.org/class/1998/> for details.
%% The ``\CCScat'' command takes four arguments.

\CCScatlist{ % not used in journal version
 \CCScat{K.6.1}{Management of Computing and Information Systems}%
{Project and People Management}{Life Cycle};
 \CCScat{K.7.m}{The Computing Profession}{Miscellaneous}{Ethics}
}

%% A teaser figure can be included as follows
\teaser{
  \centering
  \includegraphics[width=\linewidth]{CypressView}
  \caption{In the Clouds: Vancouver from Cypress Mountain. Note that the teaser may not be wider than the abstract block.}
  \label{fig:teaser}
}

%% Uncomment below to disable the manuscript note
%\renewcommand{\manuscriptnotetxt}{}

%% Copyright space is enabled by default as required by guidelines.
%% It is disabled by the 'review' option or via the following command:
% \nocopyrightspace


\vgtcinsertpkg

%%%%%%%%%%%%%%%%%%%%%%%%%%%%%%%%%%%%%%%%%%%%%%%%%%%%%%%%%%%%%%%%
%%%%%%%%%%%%%%%%%%%%%% START OF THE PAPER %%%%%%%%%%%%%%%%%%%%%%
%%%%%%%%%%%%%%%%%%%%%%%%%%%%%%%%%%%%%%%%%%%%%%%%%%%%%%%%%%%%%%%%%

\begin{document}

%% The ``\maketitle'' command must be the first command after the
%% ``\begin{document}'' command. It prepares and prints the title block.

%% the only exception to this rule is the \firstsection command

\firstsection{Introduction}
\maketitle

The distributed database system is becoming increasingly pervasive due to the explosive growth of data in science, industrial and life.
Meanwhile, many tools such as Hive, Flink and Vertical are developed to optimize the query, translate traditional query language to execution plan based on the map-reduce framework and dispatch multiple tasks to clusters to perform the data acquisition in parallel.

To maximally leverage the distributed systems, it is crucial for users to understand and evaluate how the query runs across the clusters. The frequently asked questions include "Where does the time go?", "What is the bottleneck of my query?", "Can we improve the performance of the specific query?".  Many research work devoted to evaluating and improving the performance of data analytic frameworks, but most of them try to reveal the performance by making high-level statistics about the correlated metrics collected from the accumulated logs or experiment conducted on benchmarks, which cannot be used for the understanding of the special case and provide the details answer for these questions. Specific methods which can look into the query execution process are required.

Understanding query logic and execution in the distributed environment is challenging. Three stages are involved for human-beings to understand a database query comprehensively: 1) understand the query logic, 2) understand the execution plan structure, and 3) understand the plan execution procedure.
Understand the query logic itself is long studied direction, especially when the queries are becoming increasing deep and nested.
With the pervasive of distributed database system, visual explaining of distributed execution plan structure is studied. These methods transform the abstract execution plan with thousands of lines of description to intuitive diagrams such as tree or graph and design interactions allowing users to interactively explore the nested structures.
In our paper, we mainly focus on the stage 3) i.e. understand the execution procedure of query execution.There are three challenges to facilitate the fine-grained analysis of query executions.

\stitle{Large number of atom tasks}
When issuing the execution plan on the distributed database system, large number of atomic tasks will be generated and executed on the nodes in the cluster in parallel. The duration of tasks may be significant different from each other even executed on the same server. Identifying the unnature long cases and analyzing their performance from the large amount of tasks is challenges, since it is difficult to define clear ground truth to fits all cases.

\stitle{Complex dependencies among the tasks}
The processing of a task always depends on several prerequisite tasks which provide the necessary input data. Analyzing the complex many to many dependencies and identifying the trace of interested are difficult in the large task set.

\stitle{Unpredicted behavior of distributed system} also increases he difficulty to understand the model execution procedure. For instance, the developer find that the same query execution plan run today may be different from that of yesterday. In general, four aspects are considered to affect the performance of clusters: CPU usage, memory usage, network IO and disk status. Existing work studies try to reveal how these metrics related to the system performance or quantify the impact and significance of these features. These studies are conducted based on the observed performance data from the experiment or logs collected from the production environment. One work inspired us is $\DQV$ which tries to linkage the resources status to query performance and resource usages. However, these work fail to provide the fine-grained execution traces for users to inspect the reasons of model behavior.
% There are three challenges to facilitate the fine-grained inspection of query executions. 
% \textbf{Non-transparent translation} makes it difficult for database users to inspect the query behaviors for a given abstract query. As shown by Figure**, the query issued by users is highly abstract which hides the detailed executed logic on a distributed system. The tools such as Hive can translate the query to a physical execution plan as shown by Figure~\ref{fig:exec_plan}. The execution plans have hundreds to thousands of lines of description, which is difficult for users to build a mental map for the overall execution plan. Existing work tries to bridge the gap between the execution logic and human perception by visualizing the execution plan as directed acyclic graph and allow users to interactively narrow down to any detailed operator as demand. However, the visualization of execution plan is always independent from the visualization of execution process. During the exploration, the users have to switch between multiple views which break the continuity of the plan-execution analysis.
% \textbf{Unexpected behavior of distributed system} also increases he difficulty to understand the model execution process. For instance, the developer find that the same query execution plan run today may be different from that of yesterday. In general, four aspects are considered to affect the performance of clusters: CPU usage, memory usage, network IO and disk status. Existing work studies try to reveal how these metrics related to the system performance or quantify the impact and significance of these features. These studies are conducted based on the observed performance data from the experiment or logs collected from the production environment. One work inspired us is $\DQV$ which tries to linkage the resources status to query performance and resource usages. However, these work fail to provide the fine-grained execution traces for users to inspect the reasons of model behavior.

In this work, we develop a visual analytics system called $\DQV$ (Figure~\ref{fig:teaser}) for database users to monitor, understand and diagnose query behavior across the distributed system. The system can be run with three modes: 1) monitoring mode: the system runs with the query execution process, collects and visualizes the query status in real-time; 2) simulation mode: the system will replay the execution process with given simulation rate; 3) analysis mode: the system will directly show the final results for users to explore the final results. In the visualization component, we design a temporal DAG (directed acyclic graph) diagram to display the execution plan and execution process dynamically and seamlessly. To enable the scalable visual analysis of the large number of tasks executed on the computing nodes, we implement the compound trace diagram which integrates the point cloud form and progress bar form together to meet the different analysis requirements. The monitoring results are visualized in the monitoring view and linked with the other analysis views through a suit of flexible interactions. 


%% \section{Introduction} %for journal use above \firstsection{..} instead
\begin{itemize}
% \item A framework to systematically analyze the query execution on distributed database by integrating three components: query analyzing component, machine monitoring component and analytic component.
\item The design and implementation of $\DQV$, a visual analytic system for understanding and analyzing the distributed query execution process.
\item Well-established visualization front-end to support the interactive investigating, comparing and diagnosing the query process. The system includes a set of novel designs for visualizing the temporal DAG and sequence group.
\item Case studies on the analysis of query process performed on the Hadoop platform.
\end{itemize}


\section{Using the Style Template}
abcd

\section{Bibliography Instructions}

\begin{itemize}
\item Sort all bibliographic entries alphabetically but the last name of the first author. This \LaTeX/bib\TeX\ template takes care of this sorting automatically.
\item Merge multiple references into one; e.\,g., use \cite{Max:1995:OMF,Kitware:2003} (not \cite{Kitware:2003}\cite{Max:1995:OMF}). Within each set of multiple references, the references should be sorted in ascending order. This \LaTeX/bib\TeX\ template takes care of both the merging and the sorting automatically.
\item Verify all data obtained from digital libraries, even ACM's DL and IEEE Xplore  etc.\ are sometimes wrong or incomplete.
\item Do not trust bibliographic data from other services such as Mendeley.com, Google Scholar, or similar; these are even more likely to be incorrect or incomplete.
\item Articles in journal---items to include:
  \begin{itemize}
  \item author names
	\item title
	\item journal name
	\item year
	\item volume
	\item number
	\item month of publication as variable name (i.\,e., \{jan\} for January, etc.; month ranges using \{jan \#\{/\}\# feb\} or \{jan \#\{-{}-\}\# feb\})
  \end{itemize}
\item use journal names in proper style: correct: ``IEEE Transactions on Visualization and Computer Graphics'', incorrect: ``Visualization and Computer Graphics, IEEE Transactions on''
\item Papers in proceedings---items to include:
  \begin{itemize}
  \item author names
	\item title
	\item abbreviated proceedings name: e.\,g., ``Proc.\textbackslash{} CONF\_ACRONYNM'' without the year; example: ``Proc.\textbackslash{} CHI'', ``Proc.\textbackslash{} 3DUI'', ``Proc.\textbackslash{} Eurographics'', ``Proc.\textbackslash{} EuroVis''
	\item year
	\item publisher
	\item town with country of publisher (the town can be abbreviated for well-known towns such as New York or Berlin)
  \end{itemize}
\item article/paper title convention: refrain from using curly brackets, except for acronyms/proper names/words following dashes/question marks etc.; example:
\begin{itemize}
	\item paper ``Marching Cubes: A High Resolution 3D Surface Construction Algorithm''
	\item should be entered as ``\{M\}arching \{C\}ubes: A High Resolution \{3D\} Surface Construction Algorithm'' or  ``\{M\}arching \{C\}ubes: A high resolution \{3D\} surface construction algorithm''
	\item will be typeset as ``Marching Cubes: A high resolution 3D surface construction algorithm''
\end{itemize}
\item for all entries
\begin{itemize}
	\item DOI can be entered in the DOI field as plain DOI number or as DOI url; alternative: a url in the URL field
	\item provide full page ranges AA-{}-BB
\end{itemize}
\item when citing references, do not use the reference as a sentence object; e.\,g., wrong: ``In \cite{Lorensen:1987:MCA} the authors describe \dots'', correct: ``Lorensen and Cline \cite{Lorensen:1987:MCA} describe \dots''
\end{itemize}

\section{Example Section}


\section{Exposition}

\subsection{Lorem ipsum}


\subsection{Mezcal Head}



\subsubsection{Duis Autem}




\subsubsection{Ejector Seat Reservation}




\paragraph{Confirmed Ejector Seat Reservation}


\paragraph{Rejected Ejector Seat Reservation}


\subsection{Vestibulum}

\section{Conclusion}

Conclusion


%% if specified like this the section will be committed in review mode
\acknowledgments{
The authors wish to thank A, B, and C. This work was supported in part by
a grant from XYZ (\# 12345-67890).}

%\bibliographystyle{abbrv}
\bibliographystyle{abbrv-doi}
%\bibliographystyle{abbrv-doi-narrow}
%\bibliographystyle{abbrv-doi-hyperref}
%\bibliographystyle{abbrv-doi-hyperref-narrow}

\bibliography{template}
\end{document}

