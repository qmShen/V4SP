The distributed database system is becoming increasingly pervasive due to the explosive growth of data in science, industrial and life.
Meanwhile, many tools such as Hive, Flink and Vertical are developed to optimize the query, translate traditional query language to execution plan based on the map-reduce framework and dispatch multiple tasks to clusters to perform the data acquisition in parallel.

To maximally leverage the distributed systems, it is crucial for users to understand and evaluate how the query runs across the clusters. The frequently asked questions include "Where does the time go?", "What is the bottleneck of my query?", "Can we improve the performance of the specific query?".  Many research work devoted to evaluating and improving the performance of data analytic frameworks, but most of them try to reveal the performance by making high-level statistics about the correlated metrics collected from the accumulated logs or experiment conducted on benchmarks, which cannot be used for the understanding of the special case and provide the details answer for these questions. Specific methods which can look into the query execution process are required.

Understanding query logic and execution in the distributed environment is challenging. Three stages are involved for human-beings to understand a database query comprehensively: 1) understand the query logic, 2) understand the execution plan structure, and 3) understand the plan execution procedure.
Understand the query logic itself is long studied direction, especially when the queries are becoming increasing deep and nested.
With the pervasive of distributed database system, visual explaining of distributed execution plan structure is studied. These methods transform the abstract execution plan with thousands of lines of description to intuitive diagrams such as tree or graph and design interactions allowing users to interactively explore the nested structures.
In our paper, we mainly focus on the stage 3) i.e. understand the execution procedure of query execution.There are three challenges to facilitate the fine-grained analysis of query executions.

\stitle{Large number of atom tasks}
When issuing the execution plan on the distributed database system, large number of atomic tasks will be generated and executed on the nodes in the cluster in parallel. The duration of tasks may be significant different from each other even executed on the same server. Identifying the unnature long cases and analyzing their performance from the large amount of tasks is challenges, since it is difficult to define clear ground truth to fits all cases.

\stitle{Complex dependencies among the tasks}
The processing of a task always depends on several prerequisite tasks which provide the necessary input data. Analyzing the complex many to many dependencies and identifying the trace of interested are difficult in the large task set.

\stitle{Unpredicted behavior of distributed system} also increases he difficulty to understand the model execution procedure. For instance, the developer find that the same query execution plan run today may be different from that of yesterday. In general, four aspects are considered to affect the performance of clusters: CPU usage, memory usage, network IO and disk status. Existing work studies try to reveal how these metrics related to the system performance or quantify the impact and significance of these features. These studies are conducted based on the observed performance data from the experiment or logs collected from the production environment. One work inspired us is $\DQV$ which tries to linkage the resources status to query performance and resource usages. However, these work fail to provide the fine-grained execution traces for users to inspect the reasons of model behavior.
% There are three challenges to facilitate the fine-grained inspection of query executions. 
% \textbf{Non-transparent translation} makes it difficult for database users to inspect the query behaviors for a given abstract query. As shown by Figure**, the query issued by users is highly abstract which hides the detailed executed logic on a distributed system. The tools such as Hive can translate the query to a physical execution plan as shown by Figure~\ref{fig:exec_plan}. The execution plans have hundreds to thousands of lines of description, which is difficult for users to build a mental map for the overall execution plan. Existing work tries to bridge the gap between the execution logic and human perception by visualizing the execution plan as directed acyclic graph and allow users to interactively narrow down to any detailed operator as demand. However, the visualization of execution plan is always independent from the visualization of execution process. During the exploration, the users have to switch between multiple views which break the continuity of the plan-execution analysis.
% \textbf{Unexpected behavior of distributed system} also increases he difficulty to understand the model execution process. For instance, the developer find that the same query execution plan run today may be different from that of yesterday. In general, four aspects are considered to affect the performance of clusters: CPU usage, memory usage, network IO and disk status. Existing work studies try to reveal how these metrics related to the system performance or quantify the impact and significance of these features. These studies are conducted based on the observed performance data from the experiment or logs collected from the production environment. One work inspired us is $\DQV$ which tries to linkage the resources status to query performance and resource usages. However, these work fail to provide the fine-grained execution traces for users to inspect the reasons of model behavior.

In this work, we develop a visual analytics system called $\DQV$ (Figure~\ref{fig:teaser}) for database users to monitor, understand and diagnose query behavior across the distributed system. The system can be run with three modes: 1) monitoring mode: the system runs with the query execution process, collects and visualizes the query status in real-time; 2) simulation mode: the system will replay the execution process with given simulation rate; 3) analysis mode: the system will directly show the final results for users to explore the final results. In the visualization component, we design a temporal DAG (directed acyclic graph) diagram to display the execution plan and execution process dynamically and seamlessly. To enable the scalable visual analysis of the large number of tasks executed on the computing nodes, we implement the compound trace diagram which integrates the point cloud form and progress bar form together to meet the different analysis requirements. The monitoring results are visualized in the monitoring view and linked with the other analysis views through a suit of flexible interactions. 


%% \section{Introduction} %for journal use above \firstsection{..} instead
\begin{itemize}
% \item A framework to systematically analyze the query execution on distributed database by integrating three components: query analyzing component, machine monitoring component and analytic component.
\item The design and implementation of $\DQV$, a visual analytic system for understanding and analyzing the distributed query execution process.
\item Well-established visualization front-end to support the interactive investigating, comparing and diagnosing the query process. The system includes a set of novel designs for visualizing the temporal DAG and sequence group.
\item Case studies on the analysis of query process performed on the Hadoop platform.
\end{itemize}