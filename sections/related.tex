\section{Related work}
\subsection{Query analysis}
Understanding the query behaviour and evaluting database performance has been studied for decades since the database management systems(DBMSs) have been found. Both database and visualization communities have proposed methods to analyze the performance and dignose the queires in automatic or manual way[][]. We give a brief introduction about the related works from the two following aspects: \textit{query logic structure} and \textit{query execution structure}, and refer the interested readers to~\cite{gathani2020debugging} for the systematic overview about the database query debug and performance analysis.

\emph{\textbf{Analyzing query structure}}. (SQL)Queries can be hard to read since they are always have a deep and nested structure. Many research works have been conducted to help the database users to quickly understand the quires. The most common method is to utlized visualization techniques to show the logic structure of operations[all]. For example, extended from previosu work, QueryVis utilzes the node-link diagrams to show the relationship between operators, the unambiguity is also proved in this paper. Other than the form of visualization, Gawade et al. proposed a method translate a query to Natural Language.  

\emph{\textbf{Analyzing query execution}}. 
After a query is issued by the database users, the query will be optimized and executed on the database platforms. Especially for the distributed database system, the query will be translated into the logic execution plans which is used to generating the physical tasks. Understanding the execution plans is important for users to expect the query performance. Many existing industrial sorfwares are developed to visualize the query execution process~\cite{tez-ui}. These softwares always utilize the gantt chart to show the progress and use Tree or directed acyclic graph(DAG) to show the relationship among the operators. VQA~\cite{simitsis2014vqa} displays the logic of query plan as a tree with the nodes indicating operators and the edge indicating the dataflow. Barcharts are inserted into the node to show the mertric of the operator(e.g., execution time, memory allocated). Perfopticon~\cite{moritz2015perfopticon} display the overall plan structure from two levels: fragment level and operator level. The system also allow users to observe the execution trace of fragment or operator across the workers.


\subsection{Visualization for sequence data}
Nowadays, large amount of sequence data are generated from a variety of applications such as health care~\cite{malik2015cohort, wongsuphasawat2011outflow}, social media~\cite{zhao2014fluxflow, law2018maqui}, and education~\cite{chen2015peakvizor, mu2019moocad, goulden2019ccvis, he2019vuc, chen2018viseq}.
As a special type of time-series data, event sequence record a series of discrete events in the time order of occurence~\cite{guo2020survey}. For the detail taxnomy about the time-series and event sequence visualization, we refer the readers to read the surveys~\cite{guo2020survey, silva2000visualization}. 

Sequence visualization are designed to reveal the information of event such as the event type, start time, end time and duration. Moreover, for the complex application requirement, various of the visualization tasks are proposed such visual summarization, prediction $\&$ recommendation, anormaly analysis and comparison. Existing visualization techniques can be classfied into five categories according to the form of visual representions, i.e., \emph{sankey-based visualization}, \emph{hierarchy-based visualizations}, \emph{chart-based visualizations}, \emph{timeline-based visualizations} and \emph{matrix-based visualizations}~\cite{guo2020survey}. 

\emph{\textbf{Hierarchy-based visualizations}}~\cite{gotz2019visual}, \emph{\textbf{sankey-based visualizations}} and \emph{\textbf{matrix-based visualizatios}} are always designed for displaying the sequence or sequence collection after modeling the them as special structures such as graph or tree.
For example, LifeFlow~\cite{wongsuphasawat2011lifeflow} utilizes the tree structure with a node presenting a group of events to summarize the sequences. Outflow~\cite{wongsuphasawat2011outflow} models the progression paths of sequences as directed acyclic graph with a node indicating a cluster of states, and then visualize the graph as sankey diagram~\cite{riehmann2005interactive}.  These methods always provide the highly abstract summarization for sequences and cannot directly reveal the pattern of specific individual sequence. Matrixwave~\cite{zhao2015matrixwave} uitilzes a sequence of matrix to show the connections between specific events, which can provide details connecting information of sequence. However, Matrixwave loss the detail temporal information and cannot be used to very large sequence collections. 

\emph{\textbf{Chart-based visualizations}} uses barchart, linechart or scatter plot to visualize the trend or distribution of events, which always works as assisting views to support the interactive explorations. For instance, barchart and linechart are always used to show the attributes distribution of sequences or temporal trends~\cite{gotz2019visual, cappers2017exploring}. Scatter plot can be used to show the overview of coarse-level overview of the sequence or sequence groups by project them to 2D canvas through dimension reduction algorithms or two specific attributes~\cite{wu2020visual, malik2016high, gotz2019visual}. Other than the  distribution of sequences, the scatter plot can also reveal the outliers. 

\emph{\textbf{Timeline-based visualizations}} are known as the most intuitive ways which demostrate the events in a time order. Gantt chart is a direct way to show the temporal information of event sequences, including the start time, end time and duration. Many industrial tools such as the TezViz uses gantt diagram to clearly demostrate progres of the operations. Lifeline~\cite{plaisant1996lifelines} use Gantt diagram to display the sequence as well as the events and each sequecne takes a single row. LiveGantt~\cite{jo2014livegantt} proposes an algorithm to visualize the scheduling events with better scalability. However, these methods cannot directly be used in our application since the dependencies of these sequences are ignored. Moreover, the gantt chart also suffers the series scalability problem when applied it to large sequence dataset and the abnormal sequence will be hidden without aligment.
