\section{Related work}
\subsection{Distributed query system and performance analysis}
\subsection{Visual analytics for distributed systems}
\subsection{Visualization for sequence data}
Nowadays, large amount of sequence data are generated from a variety of applications such as health care~\cite{malik2015cohort, wongsuphasawat2011outflow}, social media~\cite{zhao2014fluxflow, law2018maqui}, and education~\cite{chen2015peakvizor, mu2019moocad, goulden2019ccvis, he2019vuc, chen2018viseq}.
As a special type of time-series data, event sequence record a series of discrete events in the time order of occurence~\cite{guo2020survey}. For the detail taxnomy about the time-series and event sequence visualization, we refer the readers to read the surveys~\cite{guo2020survey, silva2000visualization}. 

Sequence visualization are designed to reveal the information of event such as the event type, start time, end time and duration. Moreover, for the complex application requirement, various of the visualization tasks are proposed such visual summarization, prediction $\&$ recommendation, anormaly analysis and comparison. Existing visualization techniques can be classfied into five categories according to the form of visual representions, i.e., \emph{sankey-based visualization}, \emph{hierarchy-based visualizations}, \emph{chart-based visualizations}, \emph{timeline-based visualizations} and \emph{matrix-based visualizatios}~\cite{guo2020survey}. 

\emph{\textbf{Hierarchy-based visualizations}}~\cite{gotz2019visual}, \emph{\textbf{sankey-based visualizations}} and \emph{\textbf{matrix-based visualizatios}} are always designed for displaying the sequence or sequence collection after modeling the them as special structures such as graph or tree.
For example, LifeFlow~\cite{wongsuphasawat2011lifeflow} utilizes the tree structure with a node presenting a group of events to summarize the sequences. Outflow~\cite{wongsuphasawat2011outflow} models the progression paths of sequences as directed acyclic graph with a node indicating a cluster of states, and then visualize the graph as sankey diagram~\cite{riehmann2005interactive}.  These methods always provide the highly abstract summarization for sequences and cannot directly reveal the pattern of specific individual sequence. Matrixwave~\cite{zhao2015matrixwave} uitilzes a sequence of matrix to show the connections between specific events, which can provide details connecting information of sequence. However, Matrixwave loss the detail temporal information and cannot be used to very large sequence collections. 

\emph{\textbf{Chart-based visualizations}} uses barchart, linechart or scatter plot to visualize the trend or distribution of events, which always works as assisting views to support the interactive explorations. For instance, barchart and linechart are always used to show the attributes distribution of sequences or temporal trends~\cite{gotz2019visual, cappers2017exploring}. Scatter plot can be used to show the overview of coarse-level overview of the sequence or sequence groups by project them to 2D canvas through dimension reduction algorithms or two specific attributes~\cite{wu2020visual, malik2016high, gotz2019visual}. Other than the  distribution of sequences, the scatter plot can also reveal the outliers. 

\emph{\textbf{Timeline-based visualizations}} are known as the most intuitive ways which demostrate the events in a time order. Gantt chart is a direct way to show the temporal information of event sequences, including the start time, end time and duration. Many industrial tools such as the TezViz uses gantt diagram to clearly demostrate progres of the operations. Lifeline~\cite{plaisant1996lifelines} use Gantt diagram to display the sequence as well as the events and each sequecne takes a single row. LiveGantt~\cite{jo2014livegantt} proposes an algorithm to visualize the scheduling events with better scalability. However, these methods cannot directly be used in our application since the dependencies of these sequences are ingored. Moreover, the gantt chart also suffers the series scalability problem when applied it to large sequence dataset and the anormal sequence will be hidden without aligment.
